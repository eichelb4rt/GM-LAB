\documentclass[sigconf, fleqn, prologue, dvipsnames]{acmart}
\usepackage{booktabs}
\usepackage{placeins}
% \usepackage{algorithmicx}
% \usepackage[noend]{algpseudocode}
% \usepackage{algorithm}
\usepackage[ruled, procnumbered]{algorithm2e}
\usepackage{subcaption}


%% \BibTeX command to typeset BibTeX logo in the docs
\AtBeginDocument{%
  \providecommand\BibTeX{{%
    \normalfont B\kern-0.5em{\scshape i\kern-0.25em b}\kern-0.8em\TeX}}}

%% These commands are for a PROCEEDINGS abstract or paper.
\settopmatter{printacmref=false} % Removes citation information below abstract
\renewcommand\footnotetextcopyrightpermission[1]{} % removes footnote with conference information in 

\acmConference[GM LAB]{Graphical Models LAB}{August 9}{Jena, Germany}


\graphicspath{{graphics/}}

\definecolor{myblue}{RGB}{46, 59, 160}
\hypersetup{
    pdfstartpage=1,
    pdfstartview = FitB,
    pdfpagelayout=SinglePage,
    pdftitle={Project Report},
    pdfsubject={Structure Learning},
    pdfauthor={Maurice Wenig},
    pdfcreator={Maurice Wenig},
    pdfproducer={Maurice Wenig},
    pdfkeywords={meta, information, pdf, hyperref, latex},
    colorlinks=true,
    linkcolor=myblue,
    citecolor=myblue
}

%----- algorithm2e
\SetKwInOut{Input}{input}
\SetKwInOut{Output}{output}

%----- new commands
\newcommand{\Romannumeral}[1]{\MakeUppercase{\romannumeral #1}}
\newcommand{\set}[1]{\{#1\}}
\newcommand{\abs}[1]{\left\vert #1 \right\vert}
\newcommand{\norm}[1]{\left\| #1 \right\|}
\newcommand{\skal}[2]{\left\langle #1 | #2 \right\rangle}
\newcommand{\numberthis}{\addtocounter{equation}{1}\tag{\theequation}}
\newcommand{\maybe}[1]{%
	\textcolor{Goldenrod}{#1?}%
    \reversemarginpar%
    \marginpar{\raggedleft\textcolor{Goldenrod}{\rule{2mm}{2mm}}}%
}
%----- defs
\def\notiff{\mathrel{{\ooalign{\hidewidth$\not\phantom{"}$\hidewidth\cr$\iff$}}}}
\def\R{\mathbb{R}}
\def\bbone{\text{\usefont{U}{bbold}{m}{n}1}}
\def\1{\mathbb{1}}
\def\O{\mathcal{O}}
\def\T{\top}
\def\pa{\text{pa}}
\def\ndy{%
    % \textcolor{red} {\hfill not done yet!}
    \reversemarginpar%
    \marginpar{\raggedleft\textcolor{red}{\rule{2mm}{2mm}}}%
}
\def\ghostline{\hfill\vspace*{-5mm}}

%----- main
\begin{document}
\title[Project Report]{Project Report\\\large Graphical Models LAB}
\author{Maurice Wenig}
\affiliation{%
	\institution{Friedrich Schiller University Jena}
	\country{Germany}%
}
\email{maurice.wenig@uni-jena.de}

\maketitle
% \let\thefootnote\relax\footnotetext{AEPRO 2022, March 1, Jena, Germany. Copyright \copyright 2022 for this paper by its authors. Use permitted under Creative Commons License Attribution 4.0 International (CC BY 4.0).}


\section{Introduction}
Learning graphical models is a big aspect of machine learning.
But for the parameters of the model to be learned, it is often assumed that the structure is already given, which includes information about dependencies and independencies between features.
However in practice, that is not the case.
Sometimes even the opposite might be true, that the structure between features itself is much more relevant than the actual amplitude of their interactions, with the hope of being able to interpret the interactions (e.g. in a causal context).
This is also an inspiration for our assignment, where our main task is coming up with a search strategy to find a sparse Bayesian Network that fits the data we were given.

For our assignment, we focused on Gaussian Bayesian Networks (GBN).
A GBN is a Bayesian Network where the conditional distributions of a feature given their parents is a Gaussian.
Another constraint to the GBN is that the mean paramater of the Gaussian, that is the conditional distribution, can only depend linearly on the values of the feature's parents in the Network Graph.
So with $n$ as the total number of features, the problem of learning the distribution $p(x_i | x_{\pa(i)})$ for a feature $i \in [n]$, its value $x_i \in \R$, and the value of their parents $x_{\pa(i)} \in \R^{n_i}$ is equivalent to learning $\hat{\beta_i}^* \in \R, \hat{\beta_i} \in \R^{n_i}, \hat{\sigma_i} \in \R$ such that $p(x_i | x_{\pa(i)}) \sim \mathcal{N}(\hat{\beta_i}^* + \hat{\beta_i}^\T x_{\pa(i)}, \hat{\sigma_i}^2)$.
Now for these parameters, the ML-estimates have a simple closed-form solution (it is just linear regression), so finding the best parameters for any given structure is simple and efficient.

% https://doi.org/10.1016/j.dss.2009.05.016
The dataset we were given spans different attributes of Portuguese "Vinho Verde" wine, which was originally used in [cite\ndy].
It consists of twelve features: eleven continuous physicochemical attributes and one discrete sensory attribute (quality).
Here, i will treat quality as a continuous feature.

\maybe{visualize data}
\FloatBarrier


\section{Methods}
% talk about what greedy search algorithm was used (in the overview paper (citation?) they said that it's rather common and well performing, so i chose the greedy algorithm (special case hill climbing with tabu walks and restarts))
% present the hill climbing algorithm (attention! details needed how tabu walks and random restarts are combined! this is not mentioned in the overview)
% -> maybe reasoning needed as to why tabu walks and the random restarts were combined in this way.
For the structure learning algorithm, i implemented a tabu search, because it generally performs well in accuracy and runtime (cite overview\ndy).
I also combined the tabu search with random restarts. The entire parameterized algorithm is illustrated in \autoref{algo:methods:search}.
It is important to note that the function arguments are all passed by reference, so mutations inside of functions will affect the passed argument even outside the function.

\begin{algorithm}
	\caption{Tabu Search with Random Restarts}
	\label{algo:methods:search}
	\Input{dataset $D$, initial DAG $G$}

	$G_{max} \gets \text{copy of } G$\;
	HillClimb($D, G_{max}$)\;
	\For{$t_0$ times}{
		TabuWalk($D, G_{max}, n_0, l$)\;
		HillClimb($D, G_{max}$)\;
	}
	\For{$t_1$ times}{
		RandomRestart($D, G_{max}, n_1$)\;
		HillClimb($D, G_{max}$)\;
		\For{$t_0$ times}{
			TabuWalk($D, G_{max}, n_0, l$)\;
			HillClimb($D, G_{max}$)\;
		}
	}
\end{algorithm}

First we do a hill climb. After the first hill climb, we do $t_0$ tabu walks followed by another hill climb each.
Then we do a random restart and do it all again. The random restart procedure is repeated $t_1$ times.
\begin{itemize}
	\item $t_0$: number of tabu walks
	\item $n_0$: maximum length of a tabu walk
	\item $l$: length of the tabu list
	\item $t_1$: number of random restarts
	\item $n_1$: number of random steps during a random restart
\end{itemize}

\FloatBarrier


\subsection{Hill Climbing}
% talk about how you climb the hill (how do you find all of the possible changes at any given point? (you need to check for cycles etc. - how do you do that efficiently (cite the guy)))
% how do you efficiently evaluate the change? (you never copy, the adjacency matrix, but apply the change, see what happens, and revert it)
% how do you evaluate if one change is better than another efficiently? (you never re-calculate the whole objective function, because the changes can be evaluated locally)
% ^ this is further discussed in [Implications for Hill Climbing]
In every iteration, the hill climbing algorithm computes the best possible elementary change among all changes.
If this change improves the objective function, it is applied.
This algorithm is illustrated in \autoref{func:methods:climb_hill}.

\begin{function}
	\caption{ClimbHill()}
	\label{func:methods:climb_hill}
	\Input{dataset $D$, current DAG $G$}
	\While{$S_{max}$ increases}{
		$S_G \gets Score(G, D)$\;
		$c_{max} \gets$ IMPOSSIBLY\_BAD\_CHANGE\;
		$C \gets AllChanges(G)$\;
		\ForEach{$c \in C$}{
			apply the change $c$ to $G$\;
			$S_c \gets Score(G, D)$\;
			undo the change $c$ in $G$\;
			\uIf{$S_c > S_{max}$}{
				$S_{max} \gets S_c$\;
				$c_{max} \gets c$\;
			}
		}
		\uIf{$S_c > S_G$}{
			apply the change $c_{max}$ to $G$\;
		}
	}
\end{function}

In order to avoid copying the adjacency matrix every time a change is applied, the adjacency matrix is changed in-place, and then the score is evaluated.
This is also improves the efficiency of generating all possible changes, which is illustrated in \autoref{func:methods:all_changes}.
The evaluation of the score can be sped up by only evaluating local changes of the changed nodes.
This is further discussed in \autoref{sec:methods:score:implications}.

\begin{function}
	\caption{AllChanges($G$)}
	\label{func:methods:all_changes}
	\Input{DAG $G$}
	\Output{all possible changes to $G$ such that $G$ is still a DAG after the application of the change}
	$C \gets \emptyset$\;
	\ForEach{$(u,v) \in E_G$}{
		$c \gets \text{FLIP}(u, v)$\;
		apply the change $c$ to $G$\;
		\uIf{$G$ does not have a cycle}{
			$C \gets C \cup \set{c}$\;
		}
		undo the change $c$ in $G$\;
	}
	\ForEach{$(u,v) \in E_G$}{
		$c \gets \text{DELETION}(u, v)$\;
		\tcp{deletions do not add cycles}
		$C \gets C \cup \set{c}$\;
	}
	$G^t \gets$ TransitiveClosure($G$)\;
	\ForEach{$u,v \in V_G, u \neq v$}{
		\uIf{$(u,v) \notin E_G$ and $(v,u) \notin E_{G^t}$}{
			$c \gets \text{ADDITION}(u, v)$\;
			$C \gets C \cup \set{c}$\;
		}
	}
	\Return{$C$}
\end{function}

Another key point to the efficiency of \autoref{func:methods:all_changes} is the efficiency of the cyclicity check.
For additions, this can be done by computing the transitive closure $G^t$ of $G$, which can be done in $\O(\abs{V_G}^3)$.
Then if $(v, u) \notin E_{G^t}$, the addition of $(u,v)$ will not create a new cycle.
For flips, i didn't find a more efficient way than completely checking the resulting graph for cycles.
This can be done in $\O(\abs{V_G} + \abs{E_G})$ with Kosaraju-Sharir's algorithm [cite\ndy].
But because we use ajdacency matrices instead of adjacency lists, Kosaraju-Sharir's algorithm takes $\O(\abs{V_G}^2)$.
Therefore, the resulting time for generating all flips is $\O(\abs{E_G} \cdot \abs{V_G}^2)$.
It follows that the overall time for generating all possible changes is $\O((\abs{V_G} + \abs{E_G}) \cdot \abs{V_G}^2)$.

\FloatBarrier


\subsection{Tabu Walks}
% talk about how you keep the tabu list efficient
% you hash the adjacency matrix to compressed byte arrays instead of saving 2-dimensional bitstrings and put em in the queue
% then at tabu walk start you put them all into set, this allows you to efficiently compute all changes that are not in the tabu lists, because lookups in the hash set are efficient. (and the number of changes you have to check is huge, so this should save a lot of time)
% maybe say that it uses the same algorithm for finding the best change out of a set of changes as the hill climbing algorithm, just that the set of changes passed is now different (non-tabu changes instead of all possible changes)
For the tabu walks, we use a FiFo-Queue that keeps track of hashes of each visited adjacency matrix. This queue is called the tabu list.
The rest of the algorithm is very similar to hill climbing, just with minimal adjustments:
\begin{itemize}
	\item changes that result in visited adjacency matrices are not considered
	\item in each iteration, the top change is applied, disregarding whether it improves the objective function
	\item if the current value of the objective function exceeds the initial value of the objective function, the tabu walk is stopped
\end{itemize}
To efficiently check which changes result in visited adjacency matrices, the tabu list was turned into a set in each iteration.


\subsection{Random Restarts}
% sometimes we just do random changes, maybe talk about your choice of uniform distribution among all possible changes (so we don't have a bias)
% maybe talk about how a bias might be introduced to prioritize additions and deletions, or maybe prioritize flips, idk might be interesting.
% priors for the random walk could be studied in future works. (maybe find out if someone already did that? cite them if someone exists -> like here, they tried but i won't get into it. if you wanna know more about it, check em out)
For the random restarts, a list of all possible changes is generated at each step.
Then one of them is chosen from a uniform distribution.


\subsection{Choice of Optimized Score}
% TODO: rewrite this? this is just for remembering what i did and why
For the score function $p(G | D) \propto p(D | G) \cdot p(G)$, i used the approximation
\begin{align*}
	p(D | G)     & \approx p(D | G, \hat{\theta})               \\
	\hat{\theta} & := \arg\max\limits_{\theta} p(D | G, \theta)
\end{align*}
and the prior
$$p(G) \propto \frac{1}{|E|^\lambda}$$
With this, the objective function can be decomposed into the sum of independent node scores and a regularization term.
\begin{align*}
	\arg\max\limits_G p(G | D) & = \arg\max\limits_G \log p(D | G) + \log p(G)                             \\
	                           & \approx \arg\max\limits_G \log p(D | G, \hat{\theta}) - \lambda \abs{E_G} \\
	                           & = \arg\max\limits_G \sum\limits_{i \in [n]} S_i(G) - \lambda \abs{E_G}
\end{align*}
where
$$S_i(G) := -\abs{D} \cdot \log \hat{\sigma_i} - \frac{1}{2} \sum\limits_{x \in D} \left(\frac{x_i - (\hat{\beta_i}^\T x_{\pa(i)} + \hat{\beta_i}^*)}{\hat{\sigma_i}}\right)^2$$
and $\hat{\theta} := \left(\hat{\beta_i}, \hat{\beta_i}^*, \hat{\sigma_i}\right)_{i \in [n]}$ are the respective ML estimates for
$$p(x_i | x_{\pa(i)}) \sim \mathcal{N}(\hat{\beta_i}^\T x_{\pa(i)} + \hat{\beta_i}^*, \sigma_i^2)$$
A derivation of this can be found in \autoref{sec:calc:score_function}.

\subsubsection{Implications for Hill Climbing}
\label{sec:methods:score:implications}
Elementary changes (addition, substraction, flip of an edge) only influence local distributions.
That means if we construct a graph $G'$, where $G'$ was made by applying an elementary change to an edge $(u, v)$ in $G$, the comparison $p(G' | D) > p(G | D)$ can be evaluated locally:
\begin{align*}
	\sum\limits_{i \in [n]} S_i(G') - \lambda \abs{E_{G'}}                               & > \sum\limits_{i \in [n]} S_i(G) - \lambda \abs{E_{G}}                  \\
	\iff\sum\limits_{i \in [n]} S_i(G') - S_i(G)                                         & > \lambda (\abs{E_{G'}} - \abs{E_{G}})                                  \\
	\iff\underbrace{\sum\limits_{i \in \set{u,v}} S_i(G') - S_i(G)}_{:= \Delta_S(G', G)} & > \lambda \underbrace{(\abs{E_{G'}} - \abs{E_{G}})}_{:= \Delta_E(G',G)}
\end{align*}
Where the second equivalence holds because $S_i(G)$ only depends on node $i$ and its parents.
Therefore $S_i(G') = S_i(G)$ for $i \notin \set{u, v}$.
Note that $\Delta_E(G',G)$ only depends on the type of change applied to $G$:
$$
	\Delta_E(G',G) = \begin{cases}
		1  & \text{addition} \\
		0  & \text{flip}     \\
		-1 & \text{deletion} \\
	\end{cases}
$$
$\Delta_S(G', G)$ measures the improvement of node scores when the change is applied to $G$.

Similarly, two alterations $G_1$ and $G_2$ of $G$ can be compared:
\begin{align*}
	\sum\limits_{i \in [n]} S_i(G_1) - \lambda \abs{E_{G_1}} & > \sum\limits_{i \in [n]} S_i(G_2) - \lambda \abs{E_{G_2}} \\
	\iff \Delta_S(G_1, G) - \Delta_S(G_2, G)                 & > \lambda \left(\Delta_E(G_1,G) - \Delta_E(G_2,G)\right)   \\
\end{align*}
A derivation of this can be found in \autoref{sec:calc:comparison}.

The interpretation of this is that a change has to bring an improvement of at least $\lambda$ per edge in order to improve the whole objective function.
But more importantly, this allow a very efficient comparison of two different changes, which we need to efficiently find the best possible change for hill climbing and tabu walks.


\subsection{Parameter Fine-Tuning}
\ndy
% how did you fine tune the parameters? what are we looking for? (for the tabu walk and the random restarts to actually climb out of local optima and result in better optima)
% include your funny little plots here


\section{Results}
\subsection{Generated Structures}
\ndy
% refer to some generated graphs for different lambdas that are visualized in the appendix
% idk look at the graphs and maybe there's something to note?
% yes there def is: how sparse is a graph for a given lambda? -> include the plot


\subsection{Performance}
% for different lambdas, sizes of tabu walks, sizes of random walks -> include specs in appendix!
Some performance analysis\ndy


\subsection{Likelihood}
\label{sec:results:likelihood}
% you gotta run this over night or something. Just specify a number of lambdas and do a double-cross validation data = (train_structure, test_structure) = (train_structure, (train_params, test_params))
% -> maybe only 3 rotations in the structure split
Cross validation time.\ndy
\begin{table}[htbp]
	\caption{Error Score Comparison}
	\label{tab:results:errors}
	\begin{tabular}{lrrr}
		\toprule
		Recommender   & RMSE  & MAE   \\
		\midrule
		user based    & 0.670 & 0.301 \\
		item based    & 0.569 & 0.222 \\
		factorization & 0.512 & 0.207 \\
		hybrid        & 0.496 & 0.193 \\
		\bottomrule
	\end{tabular}
\end{table}
\FloatBarrier


\subsubsection{Submission Scores}
\ndy
% talk about how your learned structures fucking rule on the submission website we got, especially on the sparse side.
% (maybe introduce the submission website first tho - we could anonymously submit our learned structures, and they were evaluated on an secret test dataset)


\section{Conclusion}
Some conclusion.\ndy
% i made a pretty efficient greedy search algorithm with tabu walks and random restarts that out-performs the other submitting contestants on the sparse side and is on-par on the not-so-sparse side
% (why is it efficient - because of avoiding copying adjacency matrices and efficient local comparisons of the difference in the objective function that a change brings)
% performance depends on (probably lambda, number of tabu matrices stored, length of tabu walk)
% future work could speed this up by implementing everything in faster programming languages like C++, and maybe try parallelizing the search for the best possible change.
% or maybe faster cycle checks with the context of changing one edge at a time


\typeout{}
\bibliographystyle{ACM-Reference-Format}
\bibliography{literature}\ndy


\clearpage
\appendix
\section{Calculations}
\subsection{Score Function}
\label{sec:calc:score_function}
Here we derive
$$\max\limits_G \log p(D | G, \hat{\theta}) = \max\limits_G \sum\limits_{i \in [n]} S_i(G)$$
\begin{flalign*}
	&\max\limits_G \log p(D | G, \hat{\theta})= \max\limits_G \log \left[\prod\limits_{x \in D} p(x | G, \hat{\theta})\right]&&\\
	&= \max\limits_G \log \left[\prod\limits_{x \in D} \prod\limits_{i \in [n]} p(x_i | x_{\pa(i)}, \hat{\beta_i}, \hat{\beta_i}^*, \hat{\sigma_i})\right]&&\\
	&= \max\limits_G \sum\limits_{x \in D} \sum\limits_{i \in [n]} \log p(x_i | x_{\pa(i)}, \hat{\beta_i}, \hat{\beta_i}^*, \hat{\sigma_i})&&\\
	&= \max\limits_G \sum\limits_{x \in D} \sum\limits_{i \in [n]} \log \left[\frac{1}{\sqrt{2\pi} \sigma_i} \exp \left(-\frac{1}{2}\left(\frac{x_i - (\hat{\beta_i}^\T x_{\pa(i)} + \hat{\beta_i}^*)}{\sigma_i}\right)^2\right)\right]&&\\
	&= \max\limits_G \sum\limits_{x \in D} \sum\limits_{i \in [n]} \left[-\frac{1}{2} \left(\frac{x_i - (\hat{\beta_i}^\T x_{\pa(i)} + \hat{\beta_i}^*)}{\sigma_i}\right)^2 - \log \hat{\sigma_i} - \frac{1}{2} \log 2\pi\right]&&\\
	&= \max\limits_G \sum\limits_{i \in [n]} \underbrace{\left[-\abs{D} \cdot \log \hat{\sigma_i} - \frac{1}{2} \sum\limits_{x \in D} \left(\frac{x_i - (\hat{\beta_i}^\T x_{\pa(i)} + \hat{\beta_i}^*)}{\hat{\sigma_i}}\right)^2\right]}_{= S_i(G)}\qed&&
\end{flalign*}
Note that $\hat{\theta}$ ($\hat{\beta_i}, \hat{\sigma_i}$) depends on $G$ ($\pa(i)$).

\subsection{Graph Comparison}
\label{sec:calc:comparison}
For a graph $G'$ that was made by applying one elementary change to a graph $G$, it holds that
\begin{equation}
	\sum\limits_{i \in [n]} S_i(G') = \sum\limits_{i \in [n]} S_i(G) + \Delta_S(G', G) \label{comp_1}
\end{equation}
and
\begin{equation}
	\abs{E_{G'}} = \abs{E_{G}} + \Delta_E(G', G) \label{comp_2}
\end{equation}
Therefore
\begin{align*}
	\sum\limits_{i \in [n]} S_i(G_1) - \lambda \abs{E_{G_1}}                                  & > \sum\limits_{i \in [n]} S_i(G_2) - \lambda \abs{E_{G_2}} \\
	\stackrel{\eqref{comp_1},\eqref{comp_2}}{\iff} \Delta_S(G_1, G) - \lambda \Delta_E(G_1,G) & > \Delta_S(G_2, G) - \lambda \Delta_E(G_2,G)               \\
	\iff \Delta_S(G_1, G) - \Delta_S(G_2, G)                                                  & > \lambda \left(\Delta_E(G_1,G) - \Delta_E(G_2,G)\right)   \\
\end{align*}


\end{document}
\endinput
